\documentclass[12pt,a4paper]{article}
\usepackage{media9}
\usepackage[utf8]{inputenc}
\usepackage[T1]{fontenc}
\usepackage[czech]{babel}
\usepackage{graphicx}
\usepackage{multicol}
\usepackage[left=1.8cm,text={18cm, 25cm},top=1.8cm]{geometry}
\title{Signály a systémy}
\author{Martin Pech\\xpechm00@stud.vutbr.cz}

\begin{document}
	\begin{titlepage}
		\begin{center}
			\Huge
			\textsc{Vysoké učení technické v Brně\\ Fakulta informačních technologií} \\[70mm]
			\LARGE
			Modelování a Simulace\\
			Systém hromadnné obsluhy - Hromadná osobní přeprava\\
			
			\Large Řešeno v jazyce \textbf{C++} za pomoci simulační knihovny jazyka \textbf{Simlib}\\[130mm]
			{\Large 4. prosince 2023 \hspace{108mm}Martin Pech (xpechm00)}
		\end{center}
	\end{titlepage}
	
	\section{Úvod}
	Tématem tohoto projektu je implementace abstraktního modelu autobusové linky. Projekt vznikl jako zápočtová práce do předmětu Modelování a simulace. Chování modelu bylo demonstrováno na základě existující jihomoravské autobusové linky číslo 501. Tento model si klade za cíl najít optimální počet autobusů, které je potřeba rezerovovat pro danou linku tak, aby nedocházelo k dlouhému čekání a najít optimální jízdní řád na základě zadaných parametrů o hustotě provozu a podobně. Jelikož je tento model zacílen pouze na tyto konkrétní cíle, uvažuje pouze takové jevy, které přímo ovlivňují chování modelu. Tyto jevy jsou popsány v kapitole XXX.
	
	\subsection{Autoři a původ }
\end{document}